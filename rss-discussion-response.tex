\documentclass{article}

\title{Response to comments on ``Visualization in Bayesian Workflow''}
\author{J. Gabry, D. Simpson, A. Vehtari, M. Betancourt, and A. Gelman}


\begin{document}
\maketitle

Thank you to everyone who has participated in this discussion. 

In his proposal of vote of thanks Diggle poses the following question in
response to our paper: in what sense inference can be deemed Bayesian if ``[...]
you massage your prior until it generates realisations that are concentrated (to
a greater or lesser extent) around the data [...]''?  This is a very reasonable
question, but we were \emph{not} suggesting that the prior should be manipulated
until  realizations from the prior data generating process are concentrated
around the data.  Since most readers of our paper will not be experts in
particulate matter air pollution, in order to make the comparison more accessible
we compared the prior predictive simulations of PM 2.5 concentration to the actual 
measurements we have.  But when a researcher is actually
conducting an analysis in their area of expertise, they should  have enough
familiarity with the subject matter to look at prior predictive simulations on
their own, without needing to make direct comparisons with the data that will
be used for model fitting. For example, a researcher studying PM 2.5 
levels would know that the simulated data represent concentrations that 
would be fatal to life on Earth. So our point is really that a reasonable prior is
one that yields a reasonable prior data generating process, not that the
researcher should tailor the prior to suit the particular observations in hand.

Baillie and Vandemeulebroecke say that our proposed workflow resonates with
their experience as statisticians working on drug development. We are very happy
to receive this confirmation that the workflow is being applied  successfully in
industry. Baillie and Vandemeulebroecke also point out one limitation of our
paper: there are no recommendations for how to use visualization in the final
step of communicating inferences to a more general audience. We used our limited
space to focus on visualizations intended to help people doing statistical
modeling develop better models,  but we strongly agree that translating the
inferences from those models effectively for other audiences is a vital part of
a  statistician's job. We are also thankful that Carter and other commenters
have brought attention to the challenges of  communicating to different
audiences using visualizations. This was not something we intended to address in
our paper, but  it is an important topic that deserves greater attention from
the statistics community.

\end{document}
